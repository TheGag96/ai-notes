%
%  3-intro-to-search.tex
%  Introduction To Artificial Intelligence
%
%  Created by Illya Starikov on 01/23/18.
%  Copyright 2018. Illya Starikov. All rights reserved.
%

\section{Introduction To Search}
\begin{itemize}
    \item Open-loop problem solving steps are as follows\ldots
        \begin{enumerate}
            \item Problem-formulation (actions \& states)
            \item Goal-formulation (states)
            \item Search (action sequences)
            \item Execute solution
        \end{enumerate}

        \item All well-defined problems have the following\ldots
            \begin{itemize}
                \item An initial state
                \item An actions set (usually denoted by $\textsc{Actions}(s)$)
                \item A transition model (usually denoted by $\textsc{Result}(s,\, a)$)
                \item A goal test
                \item A step cost (usually denoted by $\textsc{C}(s,\, a,\, s')$)
                \item Path cost
                \item Solution / optimal solution\footnote{The optimal solution is just the solution with the least path cost}
            \end{itemize}

        \item Search trees\ldots
            \begin{itemize}
                \item Root corresponds to initial state
                \item Search algorithms iterate through goal testing and expanding a state until goal is found
                \item Order of state expansion is critical
            \end{itemize}

        \begin{algorithmic}[1]
            \Statex{}
            \Function{Tree-Search}{problem}
                \Let{frontier}{using initial problem state}

                \State{}
                \Loop{}
                    \If{\Call{empty}{frontier}}
                        \State{\Return{Fail}}
                    \EndIf{}\State{}

                    \State{choose leaf node and remove it from frontier}\State{}

                    \If{chosen node contains goal state}
                        \State{\Return{corresponding solution}}
                    \EndIf{}

                    \State{}
                    \State{expand chosen node and add resulting nodes to frontier}
                \EndLoop{}

            \EndFunction{}
        \end{algorithmic}

        \item However, this doesn't deal with repeated students, redundant paths, or loops. For graphs, we have as follows.

        \begin{algorithmic}[1]
            \Function{Graph-Search}{problem}
                \Let{frontier}{using initial problem state}
                \Let{explored set}{$\{\}$}

                \State{}
                \Loop{}
                    \If{\Call{empty}{frontier}}
                        \State{\Return{Fail}}
                    \EndIf{}\State{}

                    \State{choose leaf node and remove it from frontier}\State{}

                    \If{chosen node contains goal state}
                        \State{\Return{corresponding solution}}
                    \EndIf{}

                    \Let{explored set}{explored set $\cup$ chosen node}

                    \State{}
                    \If{chosen node $\not\in$ frontier or explored set}
                        \State{expand chosen node and add resulting nodes to frontier}
                    \EndIf{}
                \EndLoop{}

            \EndFunction{}
        \end{algorithmic}

        \item Search node data structure
            \begin{itemize}
                \item $n.\textsc{State}$
                \item $n.\textsc{Parent-Node}$
                \item $n.\textsc{Action}$
                \item $n.\textsc{Path-Cost}$
            \end{itemize}

        \item \textbf{\textit{States are not search nodes}}
        \item We define a child nodes as follows.

        \begin{algorithmic}[1]
            \Function{Child-Node}{problem, parent, action}
                \State{\Return{node with:}}
                \Indent{}
                    \State{\textsc{State} = problem.\Call{Result}{parent.\textsc{State}, action}}
                    \State{\textsc{Parent} = parents}
                    \State{\textsc{Action} = action}
                    \State{\textsc{Path-Cost} = parents.\textsc{Path-Cost} + problem.\Call{Step-Cost}{parent.\textsc{State}, action}}
                \EndIndent{}
            \EndFunction{}
        \end{algorithmic}

    \item The frontier\ldots
        \begin{itemize}
            \item Is a set of leaf nodes
            \item Implemented as a queue with operations\ldots
                \begin{itemize}
                    \item \textsc{Empty}
                    \item \textsc{Pop}
                    \item \textsc{Insert}
                \end{itemize}

            \item Queue types: FIFO, LIFE (stack), and priority queue.
        \end{itemize}

        \item Explored set\ldots
            \begin{itemize}
                \item Set of expanded nodes
                \item Implemented typically as a hash table for constant time insertion and lookup
            \end{itemize}

        \item Problem-solving performance\ldots
            \begin{itemize}
                \item Completeness
                \item Optimality
                \item Time complexity
                \item Space complexity
            \end{itemize}

        \item Complexity in AI\ldots
            \begin{description}
                \item[$b$] Branching Factor
                \item[$d$] Depth of Shallowest Goal Node
                \item[$m$] Max Path Length in State Space
                \item[Time Complexity] Generated Nodes
                \item[Space Complexity] Max Number of Nodes Stored
                \item[Search Cost] Time + Space Complexity
                \item[Total cost] Search + Path Cost
            \end{description}
\end{itemize}
